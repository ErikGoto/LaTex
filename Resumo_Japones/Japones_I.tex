\documentclass[a4paper, 12pt]{article}
\usepackage[top=2cm, bottom=2cm, left=2.5cm, right=2.5cm]{geometry}

\usepackage[utf8]{inputenc}
\usepackage[brazilian]{babel}
\usepackage{indentfirst}

\usepackage{graphicx}
\usepackage[pdftex]{hyperref}
\graphicspath{ {imagens/} }
\usepackage{xcolor}

%Caracteres Japoneses
\usepackage{CJK}

% Definindo novas cores
\definecolor{verde}{rgb}{0.25,0.5,0.35}
\definecolor{jpurple}{rgb}{0.5,0,0.35}



\begin{document}
\begin{titlepage} %iniciando a "capa"
	\begin{center} %centralizar o texto abaixo
		{\large Unicamp}\\[0.2cm] %0,2cm é a distância entre o texto dessa linha e o texto da próxima
		{\large MC322}\\[0.2cm] % o comando \\ "manda" o texto ir para próxima linha
		{\large Prof. Esther Colombini}\\[3.2cm]
		{\bf \huge Programação Orientada a Objetos - Quiz 1}\\[5.1cm] % o comando \bf deixa o texto entre chaves em negrito. O comando \huge deixa o texto enorme
	\end{center} %término do comando centralizar
	{\large Erik Yuji Goto}\\[0.5cm] % o comando \large deixa o texto grande
	{\large RA: 234009}\\[10cm]
	\begin{center}
		{\large Campinas}\\[0.2cm]
		{\large 2020}
	\end{center}
\end{titlepage} %término da "capa"

\newpage
\section{Pronome Demonstrativo}
\subsection{kore, sore, are, dare (Aula 9)}
\begin{CJK}{UTF8}{min}

	\begin{itemize}
		\item これ - Isto(quando se refere a um objeto que esta \textit{perto} da pessoa)\\
		\item それ - Isso(quando se refere a um objeto que esta \textit{longe} da pessoa)\\
		\item あれ - Aquilo(quando se refere a um objeto que esta longe das duas pessoas)\\
		
		\item \textbf{どれ} - Qual?
		
	\end{itemize}
	
	\textbf{Exemplos:}\\
	これ\textbf{は}___です。\\
	それ\textbf{は}___です \\
	あれ\textbf{は}___です。\\
	
	\textbf{Perguntas:}\\
	これわなんです\textbf{か}\\
	
	
	
	\textbf{Objeto pertencente a alguem:}\\
	これは___の___です。\\(PESSOA) +(OBJETO \\
	
	\textbf{Este objeto pertence a quem:?}\\
	これは\textbf{だれ}のほんですか。
\end{CJK}

\subsection{kono, sono, ano (Aula 10)}
\begin{CJK}{UTF8}{min}
	\begin{itemize}
		\item この - Este(quando se refere a um objeto que esta \textit{perto} da pessoa)\\
		\item その - Esse(quando se refere a um objeto que esta \textit{longe} da pessoa)\\
		\item あの - Aquele(quando se refere a um objeto que esta longe das duas pessoas)\\
	\end{itemize}

	\textbf{Exemplo:}\\
	Esta e a caneta da Ayako sensei.\\
	このペンはあやこせんせいのです。	\\
	
	\textbf{Ao pedir algo(em contexto de compra):}\\
	をください\\
	コーラ\textbf{をください}。\\
	しんぶん\textbf{をください}。\\
	\textbf{この}Tシャツ\textbf{をください}。\\
	
\end{CJK}
	
\section{koko, soko, asoko (Aula 11)}	
\begin{CJK}{UTF8}{min}
	\begin{itemize}
		\item ここ - Aqui\\
		\item そこ - Ai\\
		\item あそこ - La/Ali\\
	\end{itemize}

	\textbf{Exemplos:}\\
	ここはゆうびんきょくです。(Aqui e o correio)\\
	田中さんはあそこです。 (O senhor Tanaka esta ali)\\
\end{CJK}

\section{Mo/tambem (Aula 12)}
\begin{CJK}{UTF8}{min}
	これ\textbf{は}1800円です。$ -> $ これ\textbf{も}1800円です。\\
	
	あのようこさ\textbf{は}にほんじんです。$ -> $ みやざきさん\textbf{も}(にほんじん)です。\\
	
	\textbf{Presente Afirmativo:}\\
	です\\
	
	\textbf{Presente Negativo:}\\
	ではありませなん\\
	
	
\end{CJK}

\section{Ne, Yo (Aula 13)}
\begin{CJK}{UTF8}{min}
	ね e usado quando o falante esta buscando a confirmaçao / acordo do ouvinte de acordo com	o que foi dito. Pode ser traduzido como:\\
	“certo?”, “ne?”.\\
	これはとんかつです\textbf{ね}。\\
	
	よ é usado quando o falante tem certeza do que está falando e está passando uma	informação nova ao ouvinte. Pode ser traduzido como: \\
	“viu”, “eu te digo”.\\
	これはとんかつです\textbf{よ}。\\
	
	と é uma partícula aditiva. É usada para somar	os objetos ou pratos que você deseja pedir.	Pode ser traduzido como: \\“e”.\\
	ラーメン\textbf{と}とんかつをください。\\
	
	
\end{CJK}

\section{Verbos de Acao (Aula 15)}
















\newpage
\section{Vocabulario}
\begin{CJK}{UTF8}{min}
\paragraph{Lugares}
	\begin{itemize}
		\item き\underline{つ}さてん/カフ\underline{エ}¥: Cafeteria
		\item ぎんこう: Banco
		\item トイレ: Banheiro
		\item としょかん: Biblioteca
		\item ゆうびんきょく: Correio
		\item がくしょく: Bandejão
		
	\end{itemize}

\paragraph{Meios de Transporte}
	\begin{itemize}
		\item てんし\underline{や}:Trem
		\item ちかてつ:Metro
		\item くるま:Carro
		\item タクシー:Taxi
		\item しんかんせん:Trem Bala
		\item ひこうき:Aviao
		\item バス:Onibus
		\item じてんじ\underline{ゃ}:Bicicleta
		\item バイク:Moto
		\item ふね:Navio
	\end{itemize}

\paragraph{Diversos}
	\begin{itemize}
		\item あさ:Manha
		\item ひる:Tarde
		\item ゆうがた:Crespusculo
		\item ばん/よる:Noite/Madrugada
		\item あさごはん:Cafe da Manha
		\item ひるごはん:Almoco
		\item ばんごはん:Janta
		\item テレビ:TV
		\item えいが:Cinema/Filme
		\item ざ\underline{つ}し:Revista
		\item おんがく:Musica
		\item スポーツ:Esporte
		\item デート:Encontro Romantico(date)
	\end{itemize}

\paragraph{Comidas/Bebidas}
	\begin{itemize}
		\item コーヒー:Cafe
		\item ジ\underline{ュ}ース:Suco
		\item みず:Agua
		\item ミルク/ぎ\underline{ゅ}うにゆう:Leite
		\item おち\underline{ゃ}:Cha
		\item おさけ:Bebida Alcoolica
		\item ハンバーガー:Hamburguer
		\item アイスクリーム:Sorvete
	\end{itemize}

\paragraph{Verbos}
	\begin{itemize}
		\item おきる\textbf{(G II)}:Acordar
		\item ねる\textbf{(G II)}:Dormir
		\item いく\textbf{(G I)}:Ir
		\item かえる\textbf{(G I)}:Chegar(em casa)
		\item くる\textbf{(G III) - きます}:Vir
		\item たべる\textbf{(G II)}:Comer
		\item のむ\textbf{(G I)}:Beber
		\item きく\textbf{(G I)}:Ouvir
		\item よむ\textbf{(G I)}:Ler
		\item はなす\textbf{(G I)}:Conversar
		\item みる\textbf{(G II)}:Assistir
		\item する\textbf{(G III) - します}:Fazer, mesmo sentido que \textit{"play"} do inglês
	\end{itemize}
\end{CJK}

\paragraph{Dias da Semana}
		\begin{CJK}{UTF8}{min}
			\begin{itemize}
				\item げつようび:Segunda
				\item かようび:Terça
				\item すいようび:Quarta
				\item もくようび:Quinta
				\item きんょうび:Sexta
				\item どようび:Sabado
				\item にちようび:Domingo
				\item しゅうまつ:Fm de semana
				
			\end{itemize}
		\end{CJK}

\paragraph{Todo[...]}
	\begin{CJK}{UTF8}{min}
		\begin{itemize}
			\item まいあさ/まいばん:Toda manha/Toda Noite
			\item まいにち:Todo dia
			\item まいし\textit{ゅ}う:Toda semana
			\item まいつき:Todo mês
			\item まいとし:Todo ano
			\item まいげつようび:Toda segunda-feira
		\end{itemize}
	\end{CJK}

\paragraph{Frequencia}
	\begin{CJK}{UTF8}{min}
		\begin{itemize}
			\item いつも:Sempre
			\item よく:Frequentemente
			\item たいてい:Geralmente
			\item ときどき:As vezes
			\item あまり:Nao muito $ \star $
			\item ぜんせん:Quase nunca $ \star $
		\end{itemize}
	$ \star $ verbo negativo
	\end{CJK}


\end{document}