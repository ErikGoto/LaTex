\documentclass[a4paper, 12pt]{article}
\usepackage[top=2cm, bottom=2cm, left=2.5cm, right=2.5cm]{geometry}
\usepackage[utf8]{inputenc}
\usepackage[brazilian]{babel}
\usepackage{indentfirst}
\usepackage{graphicx}
\usepackage[pdftex]{hyperref}
\graphicspath{ {imagens/} }
\usepackage{xcolor}
% Definindo novas cores
\definecolor{verde}{rgb}{0.25,0.5,0.35}
\definecolor{jpurple}{rgb}{0.5,0,0.35}



\begin{document}
\begin{titlepage} %iniciando a "capa"
	\begin{center} %centralizar o texto abaixo
		{\large Unicamp}\\[0.2cm] %0,2cm é a distância entre o texto dessa linha e o texto da próxima
		{\large MC322}\\[0.2cm] % o comando \\ "manda" o texto ir para próxima linha
		{\large Prof. Esther Colombini}\\[3.2cm]
		{\bf \huge Programação Orientada a Objetos - Quiz 1}\\[5.1cm] % o comando \bf deixa o texto entre chaves em negrito. O comando \huge deixa o texto enorme
	\end{center} %término do comando centralizar
	{\large Erik Yuji Goto}\\[0.5cm] % o comando \large deixa o texto grande
	{\large RA: 234009}\\[10cm]
	\begin{center}
		{\large Campinas}\\[0.2cm]
		{\large 2020}
	\end{center}
\end{titlepage} %término da "capa"

\newpage
Como o gráfico não intercepta o eixo Y exatamente no ponto 0, precisamos que uma das funções $\phi$ seja 1(ou uma constante), portanto: $ \phi_{1} = x^{0} $.\\

Além disso, percebe-se que a conta de luz tem um ligeiro crescimento ao longo dos meses. Devido a isso supomos que $ \phi_{2} = x^{1} $. O polinômio $ x^{1} $ dá característica de inclinação ao nosso gráfico.\\

Por fim, percebemos que este crescimento não é totalmente linear, por isso usamos $ x_{3} = x^{2} $. 







\end{document}