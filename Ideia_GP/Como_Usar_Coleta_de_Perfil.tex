\documentclass[a4paper, 12pt]{article}
\usepackage[top=2cm, bottom=2cm, left=2.5cm, right=2.5cm]{geometry}
\usepackage[utf8]{inputenc}
\usepackage[brazilian]{babel}
\usepackage{indentfirst}
\usepackage{graphicx}
\usepackage[pdftex]{hyperref}
\graphicspath{ {imagens/} }

\begin{document}
%
\begin{titlepage} %iniciando a "capa"
	\begin{center} %centralizar o texto abaixo
		{\large Unicamp}\\[0.2cm] %0,2cm é a distância entre o texto dessa linha e o texto da próxima
		{\large Mecatron Projetos e Consultoria Júnior}\\[0.2cm] % o comando \\ "manda" o texto ir para próxima linha
		{\large Vice-Presidência}\\[3.2cm]
		{\bf \huge USANDO AS COLETAS DE PERFIL DOS MEMBROS}\\[5.1cm] % o comando \bf deixa o texto entre chaves em negrito. O comando \huge deixa o texto enorme
	\end{center} %término do comando centralizar
	{\large Erik Yuji Goto}\\[10cm] % o comando \large deixa o texto grande
	\begin{center}
		{\large Campinas}\\[0.2cm]
		{\large 2020}
	\end{center}
\end{titlepage} %término da "capa"


\tableofcontents
\newpage

\section{Introdução}
O presente documento foi feito baseado na gestão de recursos humanos da \href{https://www.bridgewater.com/}{Bridgewater}, uma empresa americana de gestão de investimentos.\\

\begin{quote}
	\textit{"A coragem mais necessária não é aquela que o leva a prevalecer acima dos outros, mas aquela que lhe permite ser fiel ao seu eu mais verdadeiro, independentemente do que as outras pessoas queiram que você seja"}. DALIO, Ray
\end{quote}


\section{O que são Testes Psicométricos}
Naturalmente, cada membro tem experiências, crenças, e pensamentos diferentes um do outro, estes que \textbf{moldam} como cada pessoa enxerga a realidade. Por causa dessas diferenças, é essencial que saibamos e tenhamos empatia para identificar as peculiaridades individuais, e tornar a experiência a mais proveitosa possível.
Como solução para lidar com essas divergências, a Bridgewater utiliza \textbf{testes psicométricos} para entender como as pessoas pensam durante o processo seletivo, e como se comportam no trabalho. As quatro avaliações usadas são:
\begin{itemize}
	\item Tipologia de Myers-Briggs
	\item Inventário de Personalidade no Local de Trabalho
	\item Perfil de Dimensões de Equipe
	\item Teoria de Sistemas Estratificados
\end{itemize}

\textit{“Qualquer que seja o mix, todas as avaliações transmitem as preferências das pessoas para pensar e agir”}, sendo complementares entre si.
Destes testes podemos aproveitar informações úteis para que o Gestor identifique qual a atual realidade da Mecatron e medidas mais eficientes para aquele momento. \\

Se conseguirmos identificar os talentos e preferências individuais podemos \textbf{alocar} as pessoas em trabalhos nos quais elas provavelmente vão se \textbf{destacar}\\

Podemos aperfeiçoar a formação de times de projetos, que devem ser idealmente formados por pessoas com \textbf{perfis complementares}. Por exemplo, formando times com pessoas de características de \textbf{planejadoras}, e uma outra com o pensamento \textbf{flexível}; criar equipes \textbf{heterogêneas} contribuem para que a resolução de problemas dos mais diversos possam ser resolvidos mais rapidamente.
É claro que os testes psicométricos não precisam ser usados apenas para alocação de pessoas nos projetos(algo que é inviável atualmente). \\

Sabendo como uma pessoa “funciona” podemos perceber suas \textbf{inclinações} para futuros cargos de diretoria, e aproveitar isso para desenvolvê-la no \textbf{PDI}.
Em um cenário ideal, pelo menos é assim que o CEO da Bridgewater implementou junto da cultura, todos esses testes psicométricos estão disponibilizados para qualquer funcionário. A ideia por trás disso é a \textbf{transparência radical}, cria-se um ambiente de \textbf{vulnerabilidade}, pois todos tem acesso às informações “pessoais” de qualquer um; e teoricamente isso melhoraria a \textbf{relação interpessoal}, pois saberíamos, ou teríamos uma aproximação, de como relacionar com determinada pessoa dependendo do seu perfil.

\section{Testes Psicométricos}
\subsection{Inventário de Personalidade no Local de Trabalho}
É um teste derivado de dados do Departamento do Trabalho do governo americano. Sua função é \textbf{antecipar comportamentos} e ajudar a alocar melhor cada funcionário visando sua \textbf{satifação}. Por meio dele destaca-se características, como: \textit{persistência, independência, tolerância ao estresse} e \textit{pensamento analítico}.\\

\textit{"Este teste nos ajuda a compreender o que as pessoas valorizam e como farão concessões entre seus valores"}.

\subsection{Tipologia de Myers-Briggs(MBTI)}
Permite identificar diferentes tipos de personalidade. Mais especificamente a personalidade dominante em uma pessoa.\\

As 16 personalidades são as seguintes:\\
O Inventor: ENTP (Extroversão, Intuição, Racionalidade, Percepção)\\
O Arquiteto: INTP (Introversão, Intuição, Racionalidade, Percepção)\\
O Marechal: ENTJ (Extroversão, Intuição, Racionalidade, Juízo)\\
O Mentor: INTJ (Introversão, Intuição, Racionalidade, Juízo)\\
O Campeão: ENFP (Extroversão, Intuição, Emoção, Percepção)\\
O Sanador: INFP (Introversão, Intuição, Emoção, Percepção)\\
O Conselheiro: INFJ (Introversão, Intuição, Emoção, Juízo)\\
O Provedor: ESFJ (Extroversão, Sensação, Emoção, Juízo)\\
O Protetor: ISFJ (Introversão, Sensação, Emoção, Juízo)\\
O Supervisor: ESTJ (Extroversão, Sensação, Racionalidade, Juízo)\\
O Inspetor: ISTJ (Introversão, Sensação, Racionalidade, Juízo)\\
O Artista: ESFP (Extroversão, Sensação, Emoção, Percepção)\\
O Compositor: ISFP (Introversão, Sensação, Emoção, Percepção)\\
O Promotor: ESTP (Extroversão, Sensação, Racionalidade, Percepção)\\
O Artesão: ISTP (Introversión, Sensação, Racionalidade, Percepção)

\section{Algumas coisas que podemos retirar dos testes}
\textbf{Introversão x extroversão:} Introvertidos são pessoas que “tiram sua energia de ideias, memórias e experiências”, e tendem a permanecer no \textbf{mundo interior}.
Extrovertidos são pessoas que “obtêm sua energia do contato com outras pessoas”, e tendem a permanecer no \textbf{mundo exterior}. 
“Descobri que é importante ajudar a cada um a se comunicar da maneira com que se sente mais confortável”.\\

\textbf{Intuição x sensitividade:} Pensando numa excursão em uma floresta, temos dois tipos essenciais de pessoas:\\

Pessoas intuitivas têm uma visão mais \textbf{geral} de um problema, elas olham para as árvores.
Já pessoas sensitivas tem uma visão mais \textbf{detalhada}, elas olham para as folhas.
Membros intuitivos seriam indicados para áreas como o jurídico, ou contabilidade, áreas que necessitam de atenção aos detalhes.
Enquanto membros sensitivos iriam para projetos, área esta que precisa de pessoas com atenção focada no contexto, e depois nos detalhes.\\

\textbf{Pensar x sentir:} Há pessoas que tomam decisões com o \textbf{raciocínio lógico}, baseado em fatos. E outras que tomam decisões buscando a \textbf{harmonia entre as pessoas}
(simples assim kk).\\

Ao ir para o médico você espera que ele receite/diagnostique com base essencialmente em seus sintomas, e de maneira lógica.
Em contrapartida, uma pessoa trabalhando para o RH precisa ter determinadas características: empatia, contato interpessoal, construção de relacionamentos.\\

\textbf{Planejar x perceber}: \textit{"Algumas pessoas gostam de viver de um jeito planejado e ordenado, enquanto outras preferem flexibilidade e espontaneidade"}. Percebedores reagem ao ambiente e trabalham para se adaptar às mudanças, trabalhando de \textbf{fora para dentro}. Enquanto planejadores primeiro determinam um objetivo, e depois pensam o que fazer para alcançá-lo, trabalhando de \textbf{dentro para fora}.\\

\textbf{Foco nas tarefas x foco nos objetivos}: Pessoas com foco nas tarefas se concentram nas tarefas diárias, e fazem mudanças incrementais em cima de processos já existentes. Enquanto pessoas com foco nos objetivos tendem a se afastar do dia a dia e refletir sobre o andamento/progresso das coisas que fazem em direção ao objetivo.

\section{Conclusão}
O objetivo de tudo isso é \textit{"assegurar que cada integrante saiba seus pontos fracos e fortes e quais são suas responsabilidades"} para formar equipes heterogêneas de alta performance.\\

Os testes não são resultados absolutos, para complementá-los uma sugestão é realizar uma pesquisa interna semelhante ao que aconteceu no início do ano, onde as pessoas são avaliadas em relação aos seus pontos fortes e fracos por todos os outros membros que estão em convivência direta; as pessoas são muito complexas, simples testes e avaliações podem não ser suficientes para entender como elas enxergam as coisas, portanto a empatia e o contexto deve ser essencial para analisar essas informações.\\

Quero reforçar que eles não são atemporais, pois as pessoas reformulam seu modo de pensar constantemente, e esse é o intuito de fazer parte da Mecatron, já que formamos líderes empreendedores. 



	
	
\end{document}
