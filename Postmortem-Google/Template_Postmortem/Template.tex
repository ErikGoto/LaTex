\documentclass[a4paper, 12pt]{article}
\usepackage[top=2cm, bottom=2cm, left=2.5cm, right=2.5cm]{geometry}

\usepackage[utf8]{inputenc}
\usepackage[brazilian]{babel}
\usepackage{indentfirst}

\usepackage{graphicx}
\usepackage[pdftex]{hyperref}
\graphicspath{ {imagens/} }
\usepackage{xcolor}

%Caracteres Japoneses
\usepackage{CJK}

% Definindo novas cores
\definecolor{verde}{rgb}{0.25,0.5,0.35}
\definecolor{jpurple}{rgb}{0.5,0,0.35}



\begin{document}
	\Large
	Mecatron - Postmortem
	
	Autor: Erik Goto, 23/04/21\\
	Status: Em andamento\\
	\textbf{[Projeto] Robô Educativo}\\
	\large
	
	\textbf{Palavras-Chave}:
	\normalsize
	Eletrônica; ESP32; microUSB; upload \textit{[Palavras-chave para facilitar a filtragem do assunto]}
	
	\section{Gatilho}
	Refazer a placa do projeto após ela ser produzida.
	
	\section{Impacto}
	Vamos ter que refazer o circuito do micro USB para alimentação e upload do programa. Devido a isso a entrega do projeto será adiada mais algumas semanas, e "perdemos" o investimento realizado, tanto tempo quanto dinheiro.
	
	\section{Causa}
	Não testamos o circuito em uma protoboard, e apenas copiamos o CI no esquemático sem ter certeza como o circuito reagiria ao ser alimentado e numa situação de upload de código para a ESP32.
	
	\section{Plano de Ação}
	\textit{[Não sei. Mas aqui viria o plano de ação]}
	\begin{enumerate}
		\item A
		\item B
		\item C
		\item D
	\end{enumerate}
	
	
	\section{Sugestões para a posterioridade}
	\subsection{O que fizemos bem?}
	Conseguimos substituir o funcionamento do microUSB por um FTDI. Assim os outros testes não foram afetados.
	
	\subsection{O que não fizemos bem?}	
	Não testamos o comportamento dos componentes em uma protoboard antes de usá-los e encaminharmos a placa para produção.
	\subsection{Outras observações}
	\textit{[Uma das partes mais importantes do docs]}
	
	Por sorte tínhamos dois pinos para a comunicação serial que serão usados no projeto final. Portanto, o upload pôde ser realizado usando um módulo FTDI. Em um projeto que use ESP32 é importante colocar dois pinheads conectados ao RX e TX, facilitando o processo de upload, além de ser mais fácil que usar o microUSB para tal função.\\
	
	
	\textit{[Na hora da revisão feita por uma segunda pessoa, coisas que devem ser avaliadas]}
	\begin{itemize}	
		\item Was key incident data collected for posterity?
		\item Are the impact assessments complete?
		\item Was the root cause sufficiently deep?
		\item Is the action plan appropriate and are resulting bug fixes at appropriate priority?
		\item Did we share the outcome with relevant stakeholders?
	\end{itemize}
	
\end{document}