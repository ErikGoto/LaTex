\documentclass[a4paper, 12pt]{article}
\usepackage[top=2cm, bottom=2cm, left=2.5cm, right=2.5cm]{geometry}
\usepackage[utf8]{inputenc}
\usepackage[brazilian]{babel}
\usepackage{indentfirst}
\usepackage{graphicx}
\usepackage{wrapfig}
\usepackage[pdftex]{hyperref}
\graphicspath{ {imagens/} }
\usepackage{amsmath}

\begin{document}
%
\begin{titlepage} %iniciando a "capa"
	\begin{center} %centralizar o texto abaixo
		{\large Unicamp}\\[0.2cm] %0,2cm é a distância entre o texto dessa linha e o texto da próxima
		{\large Unicamp}\\[0.2cm] % o comando \\ "manda" o texto ir para próxima linha
		{\large Marco Lucio Bittencourt - Turma B}\\[3.2cm]
		{\bf \huge Dinâmica Trabalho 1}\\[0.2cm] 
		{\bf \large Movimento Relativo: Matrizes de Rotação}\\[4.9cm]
		% o comando \bf deixa o texto entre chaves em negrito. O comando \huge deixa o texto enorme
	\end{center} %término do comando centralizar
	{\large Erik Yuji Goto}\\[10cm] % o comando \large deixa o texto grande
	\begin{center}
	
		{\large Campinas}\\[0.2cm]
		{\large 2021}
	\end{center}
\end{titlepage} %término da "capa"


\tableofcontents
\newpage

\section{Matrizes de transformação de coordenadas}
	Das coordenadas móveis $B_1$ para o sistema inercial I:
	\begin{equation}
			T_\beta = \begin{bmatrix}
			cos\beta & 0 & -sin\beta\\
			0 & 1 & 0\\
			sin\beta & 0 & cos\beta
		\end{bmatrix}
	\end{equation}
	
	Das coordenadas móveis $B_1$ para o sistema móvel $B_2$:
	\begin{equation}
			T_\lambda = \begin{bmatrix}
			cos\lambda & 0 & -sin\lambda\\
			0 & 1 & 0\\
			sin\lambda & 0 & cos\lambda
		\end{bmatrix}
	\end{equation}

\section{Velocidades angulares dos sistemas móveis de referência}
	\subsection{Velocidade angular da base $B_1$ com eixo de rotação Y do sistema absoluto}
	\begin{equation}
		\omega^I_1 = 
		\begin{Bmatrix}
			0\\ \omega_1\\0\
		\end{Bmatrix}
	\end{equation}
	
	\subsection{Velocidade angular da base $B_2$ com eixo de rotação Y do sistema absoluto}
	Primeiro representamos na base móvel $B_1$
	\begin{equation}
		\omega^{B_1}_2 = 
		\begin{Bmatrix}
			0\\ \omega_2 \\0
		\end{Bmatrix}
	\end{equation}
	Transformando para as coordenadas inerciais:
	\begin{equation}
		\omega^I_2 = T^T_\beta \omega^{\beta 1}_2 = \begin{Bmatrix}
			0\\
			\omega_2\\
			0
		\end{Bmatrix}
	\end{equation}
	Agora que sabemos os valores dos vetores de velocidade angular das duas bases móveis podemos calcular a velocidade angular absoluta do sistema móvel $B_2$ ao somar os vetores $\omega_1$ e $\omega_2$:
		\begin{equation}
			\omega^I_{B_2} = \omega_1^I + \omega_2^I = \begin{Bmatrix}
			0\\
			\omega_1 + \omega_2\\
			0
		\end{Bmatrix}
		\end{equation}
		
		
\section{Acelerações angulares dos sistemas móveis de referência}
	\subsection{Aceleração angular da base $B_1$ com eixo de rotação Y do sistema absoluto}
	\begin{equation}
		\alpha^I_1 = 
		\begin{Bmatrix}
			0\\ \alpha_1\\0\
		\end{Bmatrix}
	\end{equation}
	
	\subsection{Aceleração angular da base $B_2$ com eixo de rotação Y do sistema absoluto}
	Primeiro representamos na base móvel $B_1$
	\begin{equation}
		\alpha^{B_1}_2 = 
		\begin{Bmatrix}
			0\\ \alpha_2 \\0
		\end{Bmatrix}
	\end{equation}
	Transformando para as coordenadas inerciais:
	\begin{equation}
		\alpha^I_2 = T^T_\beta \alpha^{\beta 1}_2 = \begin{Bmatrix}
			0\\
			\alpha_2\\
			0
		\end{Bmatrix}
	\end{equation}
	Agora que sabemos os valores dos vetores de aceleração angular das duas bases móveis podemos calcular a aceleração angular absoluta do sistema móvel $B_2$ ao somar os vetores $\alpha_1$ e $\alpha_2$:
		\begin{equation}
			\alpha^I_{B_2} = \alpha_1^I + \alpha_2^I = \begin{Bmatrix}
			0\\
			\alpha_1 + \alpha_2\\
			0
		\end{Bmatrix}
		\end{equation}
		

\section{Vetores posição}
	\subsection{Vetor posição entre os pontos O e C no sistema inercial}
		\begin{equation}
			r^{B_1}_{OC} = \begin{Bmatrix}
			0\\
			c + a\\
			-b 
			\end{Bmatrix}
			\Rightarrow
			r^I_{OC} = T_\beta^T r^{B_1}_{OC} = 
			\begin{Bmatrix}
				-bsin\beta\\
				c+a\\
				-bcos\beta
			\end{Bmatrix}
		\end{equation}

	\subsection{Vetor posição entre os pontos C e D no sistema inercial}
		\begin{equation}
			r^{B_2}_{CD} = \begin{Bmatrix}
				r\\
				0\\
				0
			\end{Bmatrix}\Rightarrow 
			r^I_{CD} = T^T_\beta T^T_\lambda r^{B_2}_{CD} = \begin{Bmatrix}
				rcos\lambda cos\beta + rsin \lambda sin \beta\\
				0\\
				r cos \lambda sin \beta - rsin\lambda cos \beta
			\end{Bmatrix}
		\end{equation}

		A posição do ponto D em relação ao sistema inercial será:
		\begin{equation}
			r^I_{OD} = r^I_{OC} + r^I_{CD} = \begin{Bmatrix}
				rcos\lambda cos\beta + rsin \lambda sin \beta -bsin\beta \\
				c+a\\
				r cos \lambda sin \beta - rsin\lambda cos \beta -bcos\beta
			\end{Bmatrix}
		\end{equation}

\section{Velocidade linear absoluta do ponto D}
	A velocidade linear absoluta é calculada por:
	\begin{equation}
		v_D^I = \underbrace{v_C^I}_\text{I} + \underbrace{\dot{\omega}^I_{B_2} \times r^I_{CD}}_\textbf{II} + \underbrace{v^I_{CD}}_\text{III}
 	\end{equation}
	
I. Velocidade absoluta da origem C do sistema de referência $B_2$;\\

II. Velocidade tangencial;\\

III. Velocidade relativa entre o ponto C e D, note que o vetor posição $\vec{CD}$ é constante portanto, a velocidade relativa é igua a zero. $v^I_{CD} = 0$.\\

	\textbf{Velocidade absoluta do ponto C - Origem do sistema móvel}
		\begin{equation}
			v_C^I = v_O^I + \dot{\omega}^I_1 \times r^I_{OC} + v^I_{OC}
		\end{equation}
		
		Note que, $v_O^I = 0$, pois o ponto O é o centro do sistema inercial. Além disso, $v^I_{OC} = 0$ já que $\vec{OC}$ é constante.\\
		
		Concluímos que a velocidade absoluta do ponto C é dada por:
		\begin{equation}
			v_C^I = \dot{\omega}^I_1 \times r^I_{OC} = \begin{Bmatrix}
			-\omega_1bcos\beta\\
			0\\
			\omega_1bsin\beta
			\end{Bmatrix}
		\end{equation}
		
	\textbf{Velocidade Tangencial}
	
		O vetor valocidade tangencial é dado por:
		\begin{equation}
			\dot{\omega}^I_{B_2} \times r^I_{CD} = \begin{Bmatrix}
			(\omega_1 + \omega_{B_2})r cos \lambda sin \beta - rsin\lambda cos \beta\\
			0\\
			-(\omega_1 + \omega_{B_2})rcos\lambda cos\beta + rsin \lambda sin \beta
			\end{Bmatrix}
		\end{equation}
		
		Agora que sabemos os componentes da velocidade tangencial e velocidade do ponto C conseguimos encontrar a velocidade absoluta do ponto D:
		\begin{equation}
			v_D^I = \begin{Bmatrix}
			(\omega_1 + \omega_{B_2})r cos \lambda sin \beta - rsin\lambda cos \beta -\omega_1bcos\beta \\
			0\\
			-(\omega_1 + \omega_{B_2})rcos\lambda cos\beta + rsin \lambda sin \beta + \omega_1bsin\beta
			\end{Bmatrix}
		\end{equation}
		
		
\section{Aceleração linear absoluta do ponto D}
	A aceleração linear absoluta é calculada por:
	\begin{equation}
		a^I_D = a^I_C + \dot{\omega}^I_{B_2} \times \dot{\omega}^I_{B_2} \times r^I_{CD} + \ddot{\omega}^I_{B_2} \times r^I_{CD} + 2 \dot{\omega}^I_{B_2} \times v^I_{CD} + a^I_{CD}
	\end{equation}
		
	Podemos simplificar a equação acima:
		\begin{itemize}
			\item O vetor posição $\vec{CD}$ é constante
				\begin{itemize}
					\item $v^I_{CD} = 0$
					\item $a^I_{CD} = 0$
				\end{itemize} 						
		\end{itemize}
		
		A equação final para a aceleração aplicada ao nosso problema fica:
		\begin{equation}
			a^I_D = \underbrace{a^I_C}_\text{I} + \underbrace{\dot{\omega}^I_{B_2} \times \dot{\omega}^I_{B_2} \times r^I_{CD}}_\text{II} + \underbrace{\ddot{\omega}^I_{B_2} \times r^I_{CD}}_\text{III}
		\end{equation}
		
		I. Aceleração absoluta do ponto C\\
		
		II. Aceleração normal do ponto D\\
		
		III. Aceleração tangencial\\
		
		\textbf{Aceleração absoluta do ponto C}
		
		\begin{equation}
			a^I_C = a^I_O + \dot{\omega}^I_1 \times \dot{\omega}^I_1 \times r^I_{OC} + \ddot{\omega}^I_1 \times r^I_{OC} + 2 \dot{\omega}^I_1 \times v^I_{OC} + a^I_{OC}
		\end{equation}
		
		Para simplificar a equação acima vamos usar como hipósteses:
		\begin{itemize}
			\item O vetor posição $\vec{OC}$ é constante
				\begin{itemize}
					\item $v^I_{OC} = 0$	
					\item $a^I_{OC} = 0$
				\end{itemize}
			\item O ponto O é origem do sistema inercial
				\begin{itemize}
					\item $a^I_{O} = 0$	
				\end{itemize}								
		\end{itemize}				
		Portanto, a aceleração do ponto C é dada por:
		\begin{equation}
			a^I_C = \dot{\omega}^I_1 \times \dot{\omega}^I_1 \times r^I_{OC} + \ddot{\omega}^I_1 \times r^I_{OC}= \begin{Bmatrix}
			\omega^2_1 bsin\beta -\alpha_1 bcos\beta\\
			0\\
			\omega^2_1 b cos \beta + \alpha_1 b sen \beta
			\end{Bmatrix} 
		\end{equation}
		
		\textbf{Aceleração Normal do ponto D}
			\begin{equation}
				\dot{\omega}^I_{B_2} \times \dot{\omega}^I_{B_2} \times r^I_{CD} = \begin{Bmatrix}
					-(\omega_1 + \omega_2)^2(rcos \lambda cos \beta + r sen \lambda sen \beta)\\
					0\\
					-(\omega_1 + \omega_2)^2(r cos \lambda sen \beta - r sen \lambda cos \beta)
				\end{Bmatrix}
			\end{equation}
			
		\textbf{Aceleração Tangencial do ponto D}
			\begin{equation}
				\ddot{\omega}^I_{B_2} \times r^I_{CD} = 
				\begin{Bmatrix}
					(\alpha_1 + \alpha_2)(r cos \lambda sen \beta - r sen \lambda cos \beta)\\
					0\\
					-(\alpha_1 + \alpha_2)(rcos \lambda cos \beta + r sen \lambda sen \beta)
				\end{Bmatrix}
			\end{equation}
		
		
		
		
		
		
		
		
		
		
		
		
		
		
		
		
		
		
		
		
		
		
		





\end{document}
